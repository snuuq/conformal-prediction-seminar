%----------------------------------------------------------------------------
% Unofficial LaTex beamer theme for Seoul National University
%
% 
%----------------------------------------------------------------------------

\documentclass[compress]{beamer}
\usepackage{beamerthemeshadow}
\usetheme{snubeam}
\usepackage{snucode}

% Min and max
\usepackage{amssymb,amsmath,amsthm,bbm}

\newcommand{\argmin}{\mathop{\mathrm{argmin}}}
\newcommand{\argmax}{\mathop{\mathrm{argmax}}}
\newcommand{\minimize}{\mathop{\mathrm{minimize}}}
\newcommand{\maximize}{\mathop{\mathrm{maximize}}}
\newcommand{\st}{\mathop{\mathrm{subject\,\,to}}}

%shortcuts
\newcommand{\err}{\mathrm{err}}
\newcommand{\quantile}{\mathrm{Quantile}}

\newcommand{\eqd}{\stackrel{\textnormal{d}}{=}}
\newcommand{\dto}{\stackrel{\textnormal{d}}{\to}}
\newcommand{\asto}{\stackrel{\textnormal{a.s.}}{\to}}
\newcommand{\aseq}{\stackrel{\textnormal{a.s.}}{=}}
\newcommand{\asgeq}{\stackrel{\textnormal{a.s.}}{\geq}}
\newcommand{\asleq}{\stackrel{\textnormal{a.s.}}{\leq}}
\newcommand{\iidsim}{\stackrel{\textnormal{i.i.d.}}{\sim}}
\newcommand{\swap}{\mathrm{swap}}
\newcommand{\dtv}{\textnormal{d}_{\textnormal{TV}}}
% Shortcuts
\def\R{\mathbb{R}}
\def\C{\mathbb{C}}
\def\Z{\mathbb{Z}}
\def\N{\mathbb{N}}
\def\E{\mathbb{E}}
\def\P{\mathbb{P}}
\def\T{\mathsf{T}}
\def\Cov{\mathrm{Cov}}
\def\Var{\mathrm{Var}}
\def\indep{\perp\!\!\!\perp}
\def\th{^{\text{th}}}
\def\tr{\mathrm{tr}}
\def\df{\mathrm{df}}
\def\dim{\mathrm{dim}}
\def\col{\mathrm{col}}
\def\row{\mathrm{row}}
\def\nul{\mathrm{null}}
\def\rank{\mathrm{rank}}
\def\nuli{\mathrm{nullity}}
\def\spa{\mathrm{span}}
\def\sign{\mathrm{sign}}
\def\supp{\mathrm{supp}}
\def\diag{\mathrm{diag}}
\def\aff{\mathrm{aff}}
\def\conv{\mathrm{conv}}
\def\dom{\mathrm{dom}}
\def\hy{\hat{y}}
\def\hf{\hat{f}}
\def\hmu{\hat{\mu}}
\def\halpha{\hat{\alpha}}
\def\hbeta{\hat{\beta}}
\def\htheta{\hat{\theta}}
\def\cA{\mathcal{A}}
\def\cB{\mathcal{B}}
\def\cC{\mathcal{C}}
\def\cD{\mathcal{D}}
\def\cE{\mathcal{E}}
\def\cF{\mathcal{F}}
\def\cG{\mathcal{G}}
\def\cK{\mathcal{K}}
\def\cH{\mathcal{H}}
\def\cI{\mathcal{I}}
\def\cL{\mathcal{L}}
\def\cM{\mathcal{M}}
\def\cN{\mathcal{N}}
\def\cP{\mathcal{P}}
\def\cR{\mathcal{R}}
\def\cS{\mathcal{S}}
\def\cT{\mathcal{T}}
\def\cW{\mathcal{W}}
\def\cX{\mathcal{X}}
\def\cY{\mathcal{Y}}
\def\cZ{\mathcal{Z}}
\def\lac{\mathrm{LAC}}

\def\bfW{\mathbf{W}}
\def\bfX{\mathbf{X}}
\def\bfY{\mathbf{Y}}
\def\bfw{\mathbf{w}}
\def\bfx{\mathbf{x}}
\def\bfy{\mathbf{y}}

\def\bigo{\mathcal{O}}
\def\littleo{\mathrm{o}}
\def\fpre{f_{\mathrm{pre}}}

% Math shortcuts
\def\abs#1{\left|#1\right|}
\def\ind#1{\mathbbm{1}\left\{#1\right\}}
\def\indsub#1{\mathbbm{1}_{#1}}
\usepackage{mathtools}
\DeclarePairedDelimiter{\ceil}{\lceil}{\rceil}


\begin{document}
\title{Chap. 4: Conditional Coverage}
\author{Sunghee Park}
\institute{
	Dept. of Statistics\\
    Seoul National University}
\date{September 16, 2025}

{
% \usebackgroundtemplate{\includegraphics[width=\paperwidth]{images/snu-shift}}
\begin{frame}[plain]
  \titlepage
\end{frame}
}

\begin{frame}\frametitle{Table of contents}
    \tableofcontents
\end{frame}

\section{Introduction}
\begin{frame}\frametitle{Conditional Coverage of the Conformal Prediction}
	\begin{itemize}
		\item We have already guaranteed the marginal coverage of the conformal prediction.
		\begin{align*}
			1-\alpha \le \mathbb{P}(Y_{n+1}\in \mathcal{C}(X_{n+1}))
		\end{align*}
		\item This chapter focuses on the conditional coverage, which is $\mathbb{P}(Y_{n+1}\in \mathcal{C}(X_{n+1})|W)$ and satisfies the equation below:
		\begin{align*}
			\mathbb{E}[\mathbb{P}(Y_{n+1}\in \mathcal{C}(X_{n+1})|W)] = \mathbb{P}(Y_{n+1}\in \mathcal{C}(X_{n+1}))
		\end{align*}
		
		\item However, that relationship doesn't guarantee 
		\begin{align*}
			1-\alpha \le \mathbb{P}(Y_{n+1}\in \mathcal{C}(X_{n+1})) \Rightarrow 1-\alpha \le \mathbb{P}(Y_{n+1}\in \mathcal{C}(X_{n+1})|W)
		\end{align*}
		\textbf{AT ALL.}
	\end{itemize}
\end{frame}

\begin{frame}\frametitle{What this chapter will cover}
	Recall that the conditional coverage is $\mathbb{P}(Y_{n+1}\in \mathcal{C}(X_{n+1})|W)$.\\
	This chapter will cover	
	\begin{itemize}
		\item Training-conditional coverage ($W$ is related to $\mathcal{D}$)
		\item Test-conditional coverage ($W$ is related to $X_{n+1}$)
		\item and others...
	\end{itemize}
\end{frame}

\section{Training-conditional coverage}
\begin{frame}\frametitle{Goals and Assumptions}
	The goal is to make $\mathbb{P}(Y_{n+1}\in \mathcal{C}(X_{n+1})|\cD) > 1-\alpha$ almost surely.\\

	For some theoretical support of the conditional coverage, we need to make strong assumptions.
	\begin{itemize}
		\item We assume split conformal prediction. Which means that the score $s((x, y); \mathcal{D})$ and $\mathcal{D}$ are independent.
		\item The data points are i.i.d
	\end{itemize}
	If then, we can conclude Theorem 4.1 below.
	\begin{block}{Theorem 4.1. Distribution of the training conditional coverage}
		With previous assumptions,
		\begin{align*}
			&\mathbb{P}\left( 
			\mathbb{P}(Y_{n+1} \in \mathcal{C}(X_{n+1}) \mid \mathcal{D}_n) 
			\leq 1 - \alpha - \Delta 
			\right) 
			\\&\leq F_{\text{Beta}((1-\alpha)(n+1),\, \alpha(n+1))}(1 - \alpha - \Delta) 
			\leq e^{-2n\Delta^2},
		\end{align*}
	\end{block}
\end{frame}

\begin{frame}\frametitle{Sketch of the proof of Theorem 4.1}
	\begin{itemize}
		\item Let $s(X, Y) \sim F$ (CDF), $(X, Y)\sim P$, $S_i \sim F$ (i.i.d)
		\item By the algorihm of the CP, we can induce that
		\begin{align*}
			Y_{n+1}\in \mathcal{C}(X_{n+1}) \Leftrightarrow S_{n+1}\le S_{(k)} ~~(k = \lceil(1-\alpha)(n+1)\rceil)
		\end{align*}
		and
		\begin{align*}
			\P(Y_{n+1}\in\cC(X_{n+1})\mid \cD_n) = \P(S_{n+1}\leq S_{(k)}\mid \cD_n) = F(S_{(k)})
		\end{align*}
		\item Since $S_{i}\sim F$, by using the basic property of CDF (sim. as Probability integral transform), $F(S_i) \sim U_i$ and F preserves the order, (monotone increase)
		$F(S_{(i)}) \sim U_{(i)}$,\newline
		Let $U_i \sim U[0, 1]$ (i.i.d)
		\begin{align*}
			\P\big(F(S_{(k)})\leq 1-\alpha-\Delta\big) = \P\big(U_{(k)} \leq 1-\alpha-\Delta\big) \\
			\leq \P\big(U^*_{(k)}\leq 1-\alpha-\Delta\big) = F_{\text{Beta}((1-\alpha)(n+1),\, \alpha(n+1))}(1 - \alpha - \Delta)
		\end{align*}
		holds since the property of Beta distribution. 
	\end{itemize}
\end{frame}

\begin{frame}\frametitle{The meaning of Theorem 4.1}
	\begin{block}{Theorem 4.1. Distribution of the training conditional coverage}
		With previous assumptions,
		\begin{align*}
			&\mathbb{P}\left( 
			\mathbb{P}(Y_{n+1} \in \mathcal{C}(X_{n+1}) \mid \mathcal{D}_n) 
			\leq 1 - \alpha - \Delta 
			\right) 
			\\&\leq F_{\text{Beta}((1-\alpha)(n+1),\, \alpha(n+1))}(1 - \alpha - \Delta) 
			\leq e^{-2n\Delta^2},
		\end{align*}
	\end{block}
	\begin{itemize}
		\item This theorem implies when we choose $\alpha'$ for CP, which is 'stricter' then $\alpha$, we can make
		\begin{align*}
			\P\left(\P\left(Y_{n+1} \not\in \cC\left(X_{n+1}\right) \big| \: \cD_n\right)\le \alpha \ \right) \ge 1- \delta
		\end{align*}
		\item In detail, we can choose $\alpha'$ as:
		\begin{align*}
			F_{\textnormal{Beta}((1-\alpha')(n+1),\alpha'(n+1))}(1-\alpha)  = \delta
		\end{align*}
		\item It no longer guarantee in only the 'exchangeable' data (c.f. marginal case) \\Since the beta-dist. was from iid of $(X, y) \rightarrow \{S_i\} \rightarrow \{U_i\} $ 
	\end{itemize}
\end{frame}



\begin{frame}\frametitle{Hardness result for training-conditional coverage}
	Actually, it is impossible to fully guarantee training
	conditional coverage for \underline{full conformal prediction}
	\begin{block}{Theorem 4.3. Hardness result for training-conditional coverage}
		Let $P$ be any distribution on $\mathcal{X\times Y}$ s.t. $P_x$ is nonatomic*, there exists symmetric conformal score ftn $s$ s.t. when running full conformal prediction with this choice
 of $s$,
		\begin{align*}
			\mathbb{P}\left(\mathbb{P}(Y_{n+1}\in \mathcal{C}(X_{n+1})|\mathcal{D}_n) = 0\right)\ge \alpha - O\left(\sqrt{\dfrac{\log n}{n}}\right)
		\end{align*}
		where the probability is taken with respect to the training set $\cD_n=((X_1,Y_1),\dots,(X_n,Y_n))$ drawn i.i.d.\ from $P$.
	\end{block}
	*nonatomic $\Leftrightarrow \text{atom}(P) = \{z|\P_{P}(z)>0\} = \emptyset$
\end{frame}

\begin{frame}\frametitle{Sketch of the proof of Theorem 4.3}
	\begin{itemize}
		\item Constructive Proof
		\item Since $P$ is nonatomic, $\exists a : \cX \rightarrow \{0, 1, ..., n - 1\}$ with $a(X)$ has a equal prob. in domain when $X\sim P$
		\item For dataset $\cD = ((x_1,y_1),\dots,(x_k,y_k))$ and an additional data point $(x,y)$, Consider
		\begin{align*}
			s\big((x,y);\cD\big) = \ind{\textnormal{mod}\left( -a(x) + \sum_{j=1}^k a(x_j)  , n \right) < N}
		\end{align*}
		\item This score has quite useful properties in the proof:
		\begin{itemize}  
			\item Its domain is $\{0, 1\}$ - Simply think coverage fail only $S_{n+1}=1 \land \hat{q} = 0 \Rightarrow \alpha_P (\cD_n ) = \P(S_{n+1} = 1, \hat{q}=0|\cD_n)$
			\item $S_{n+1} = s\big((X_{n+1},Y_{n+1});\cD_{n+1}\big) = \ind{\textnormal{mod}\left(  \sum_{i=1}^n a(X_i)  , n \right) < N}$ (ftn unrelated to test point)
		\end{itemize}
		\item Define $\cE_{\textnormal{mod}}$ as the event that $\textnormal{mod}\left(\sum_{i=1}^n a(X_i)  , n \right) < N$. 
	\end{itemize}
\end{frame}

\begin{frame}\frametitle{Sketch of the proof of Theorem 4.3 (Cont')}
	\begin{itemize}
		\item Let $\alpha_P(\cD_n) := \P(Y_{n+1} \notin \cC(X_{n+1}) \mid \cD_n) = \indsub{\cE_{\textnormal{mod}}}\cdot \P\left( \quantile\left(S_1,\dots,S_{n+1};1-\alpha\right)=0\mid \cD_n\right)$ 
		\item Using Sliding Window Method : Let $W_k = \big\{i\in \{0,\dots,n-1\} : \textnormal{mod}(-i + k - 1 , n )\geq N\big\}$, let $\cE_{\textnormal{unif}}$ be the event that
\begin{equation*}\sum_{i=1}^n\ind{a(X_i)\in W_k} \geq (1-\alpha)(n+1)\textnormal{ for all integers $k$},\end{equation*}
i.e., each window of indices $W_k$ contains a sufficient fraction of the sample.
		\item $S_i = \ind{a(X_i)\notin W_{1 + \sum_j a(X_j)}}$ and by the property of $\cE_{\textnormal{unif}}$.
		\small
		\begin{align*}
			\P\left(\alpha_P(\cD_n)=1\right) \geq \P\left(\cE_{\textnormal{mod}}\cap \cE_{\textnormal{unif}}\right) \geq \P\left(\cE_{\textnormal{mod}}\right) -  \P\left(\cE_{\textnormal{unif}}^c\right)= \frac{N}{n} - \P\left(\cE_{\textnormal{unif}}^c\right)
		\end{align*}
		\normalsize
		\item Make upper bound of $\P(\cE_{\textnormal{mod}})$ using tail-prob. of Binomial dist. and set $N = \alpha n - O(n \log n)$
	\end{itemize}
\end{frame}

\section{Test-conditional coverage}
\begin{frame}
	\frametitle{Goals}
	The goal is to make $\mathbb{P}(Y_{n+1}\in \mathcal{C}(X_{n+1})|X_{n+1}) > 1-\alpha$ almost surely.\\ 
\end{frame}

\begin{frame}\frametitle{CP satisfying conditional coverage on discrete $X$}
	
	\begin{alertblock}{CP satisfying conditional coverage on discrete $X$}
		\begin{itemize}
		\item Suppose that  $\cX = \{x_1, x_2, ..., x_{|\cX|}\}$ then 
		 $\cC(x_k) = \{y : s(x_k, y)\le \hat{q_k}\}$ where
			$\hat{q}_k = \quantile\left((S_i)_{i\in [n], X_i = x_k} ; (1-\alpha)(1+1/n_k)\right)$\\
		\item Similar to the naive CP, but now, we choose the quantile in a smaller group which has same $X$ value.
		\end{itemize}
	\end{alertblock}
	\begin{itemize}
		\item This will make the similar result to the naive CP when the dist. of $s(X, Y)|X=x_i$ are similar.
		\item The detailed proof will be discussed later.
	\end{itemize}
\end{frame}

\begin{frame}\frametitle{CP satisfying conditional coverage on continuous $X$}
	\begin{itemize}
		\item Actually, there is no "nice" prediction method which satisfying the conditional coverage on continuous $X$ almost surely.
		\begin{block}{Theorem 4.4.}
			Suppose $\cC$ is any procedure  that satisfies distribution-free conditional coverage, i.e., for any distribution $P$ on $\cX\times\cY$, 
			$\P\big(Y_{n+1}\in\cC(X_{n+1})\mid X_{n+1}\big)\geq 1-\alpha$
			holds almost surely, \\
			where the probability is taken with respect to $(X_1,Y_1),\cdots,(X_{n+1},Y_{n+1})\iidsim P$.
			\\Then, for any distribution $P$ on $\cX\times\cY$ for which the marginal $P_X$ is nonatomic, 
			$\P\big(y\in \cC(x)\big)\geq 1-\alpha\textnormal{ for every $(x,y)\in\cX\times\cY$}$
		\end{block}
	\end{itemize}
\end{frame}

\begin{frame}\frametitle{Sketch of the proof of Theorem 4.4}
	\begin{itemize}
		\item For all $(x, y)\in\cX\times\cY$ and $\epsilon>0$\\
		\item Using "Distridution-Free" feature of $\cC$, Define a mixture dist.
		$P' = (1-\epsilon)P+\epsilon\delta_{(x, y)}$\\
		\item Since $\P_{P'}(X_{n+1}=x) > 0$, by assumption, $\P_{P'}\big(Y_{n+1}\in\cC(X_{n+1})\mid X_{n+1}=x\big)\geq 1-\alpha$
		\item Moreover, $P_X$ is nonatomic from the assumption,\\ 
		$\P_{P'}(Y_{n+1}=y|X_{n+1}=x) = 1$ and
		$\P_{P'}\big(y\in\cC(X_{n+1})\mid X_{n+1} = x\big) = \P_{P'}\big(y\in\cC(x)\big) \geq 1-\alpha$
		\item We can conclude
		\\$\P_{P}\big(y\in\cC(x)\big)\ge \P_{P'}\big(y\in\cC(x)\big)-d_{TV}(P^n, P'^{n})\ge 1-\alpha-n\epsilon$
	\end{itemize}
\end{frame}

\begin{frame}\frametitle{CP satisfying conditional coverage on continuous $X$ (Cont.)}
	\begin{block}{Corollary 4.5. Lebesgue measure limitations of the continuous case.}
		Suppose $\cC$ is any procedure that satisfies distribution-free conditional coverage, i.e., for any distribution $P$ on $\cX\times\cY$,
		\begin{align*}
			\P\big(\textnormal{Leb}(\cC(x)) = \infty\big)\geq 1-\alpha
		\end{align*}
		where Leb(·) denotes the Lebesgue measure.
	\end{block}
\end{frame}

\begin{frame}\frametitle{Proof of Corollary 4.5.}
	\begin{itemize}
		\item By definition of Lebesque measure,
		\small
		\begin{align*}
			\textnormal{Leb}(\cC(x)) &= \int_{\mathbb{R}} \ind{y\in\cC(x)} dy\leq a \Longrightarrow \int_{y=0}^{a+b}\ind{y\in\cC(x)}\;\mathsf{d}y\leq a \\
			&\Longleftrightarrow \int_{y=0}^{a+b}\ind{y\not\in\cC(x)}\;\mathsf{d}y\geq b.
		\end{align*}
		\normalsize
		\item Apply Markov's Inequality and Fubini's thm,
		\small
		\begin{align*}
			\P\left(\textnormal{Leb}(\cC(x))\leq a\right) \leq \P\left(\int_{y=0}^{a+b}\ind{y\not\in\cC(x)}\;\mathsf{d}y\geq b\right) \\
			 \leq \frac{\E\left[\int_{y=0}^{a+b}\ind{y\not\in\cC(x)}\;\mathsf{d}y\right]}{b} = \frac{\int_{y=0}^{a+b}\P(y\not\in\cC(x))\;\mathsf{d}y}{b} \leq \frac{(a+b)\alpha}{b}
		\end{align*}
		\normalsize
		\item Take $b\rightarrow \infty$ and $a\rightarrow \infty$ in order
	\end{itemize}
\end{frame}

\section{Relaxation Approach on test-conditional coverage}
\begin{frame}\frametitle{Relaxation Approach on test-conditional coverage}
	\begin{itemize}
		\item So far, we concluded that it is impossible to achieve pointwise test-conditional coverage in continuous setting.
		\item One of the idea is relaxing the problem to discrete version - which is easily available.
		\item New goal:
		$1-\alpha \le \mathbb{P}(Y_{n+1}\in \mathcal{C}(X_{n+1})|\cX_k)$ for $k \in [K]]$ with $P(\cX_k) > 0$
	\end{itemize}

	\begin{alertblock}{CP satisfying new goal (Binned Conditional Coverage)}
		\begin{itemize}
			\item $\cC(X_{n+1}) = \{y : s(X_{n+1}, y)\le \hat{q_{k(X_{n+1})}}\}$ where
				$\hat{q}_k = \quantile\left((S_i)_{i\in [n], X_i \in \cX_k} ; (1-\alpha)(1+1/n_k)\right)$\\
			\item Similar to the naive CP, but now, we choose group-(score) quantile $\cX_i$
		\end{itemize}
	\end{alertblock}
\end{frame}

\begin{frame}\frametitle{Label-conditional coverage}
	\begin{itemize}
		\item Goal : to make $\mathbb{P}(Y_{n+1}\in \mathcal{C}(X_{n+1})|Y_{n+1}) > 1-\alpha$ almost surely
	\end{itemize}
	\begin{alertblock}{CP satisfying the goal (Label-conditional coverage)}
		\begin{itemize}
			\item $\cC(X_{n+1}) = \{y : S_{n+1}^y \le \hat{q}^y\}$ where $\hat{q}^y = \quantile\left((S_i)_{i\in{\cI_y}} ; (1-\alpha)(1+1/|\cI_y|)\right),\ \cI_y = \{i\in[n]:Y_i = y\}$
			\item Similar to the naive CP, but now, we group by y
		\end{itemize}
	\end{alertblock}
\end{frame}

\begin{frame}\frametitle{Mondrian CP : Generalized CP}
	\begin{itemize}
		\item Binned Conditional Coverage ($g(x, y) = k~(x\in\cX_k)$) and Label-conditional coverage ($g(x, y) = y$) are the special cases.
		\item General Goals : to make $\mathbb{P}(Y_{n+1}\in \mathcal{C}(X_{n+1})|g(X_{n+1}, Y_{n+1})) \ge 1-\alpha$ w.p.1
	\end{itemize}
	\begin{alertblock}{Mondrian Conformal Prediction}
		\begin{itemize}
			\item $\cC(X_{n+1}) = \{y : S_{n+1}^y \le \hat{q}^y\}$ where $\hat{q}^y = \quantile\left((S_i)_{i\in{\cI_{g(X_{n+1}, y)}}} ; (1-\alpha)(1+1/|\cI_{g(X_{n+1}, y)}|)\right)$, $\cI_k = \{i\in[n]:g(X_i, Y_i) = k\}$
			\item CP with 'Grouping'
		\end{itemize}
	\end{alertblock}
\end{frame}

\begin{frame}\frametitle{Proof of the validity of Mondrian CP}
	\begin{itemize}
		\item First, we should show the statement below holds.
	\end{itemize}
	\begin{block}{Lemma 4.7. Conditional exchangeability within a bin}
		Suppose $(X_1,Y_1),\dots,(X_{n+1},Y_{n+1})$ are exchangeable. Fix any subset $\cZ_0\subseteq\cX\times\cY$, and for any fixed nonempty subset $I\subseteq[n+1]$, let $\cE_I$ be the event that $\{i \in[n+1]: (X_i,Y_i)\in\cZ_0\} = I$. If $\cE_I$ has positive probability, then $((X_i,Y_i))_{i\in I}$ is exchangeable conditional on $\cE_I$.
	\end{block}
	\begin{itemize}
		\item The key of the proof is that for arbitrary $\sigma \in perm(I)$, think the extended permutation $\tilde{\sigma} \in perm([n+1])$ which satisfies the following equation and apply the exchangeablilty in $[n+1]$ wisely to show $\P((Z_i)_{i\in I}\in A, \ \cE_I) =\P((Z_{\sigma(i)})_{i\in I}\in A, \ \cE_I)$
		\begin{align*}
			\tilde{\sigma}(i) = \begin{cases}
				\sigma(i)&i\in I\\
				i&i\notin I
			\end{cases}
		\end{align*}
	\end{itemize}
\end{frame}



\begin{frame}\frametitle{Proof of the validity of Mondrian CP (Cont')}
	\begin{itemize}
		\item Then, we prove the main theorem.
		\item First, by construction of the set $\cC(X_{n+1})$, we can see that for any $k\in[K]$, on the event $g(X_{n+1},Y_{n+1})=k$, 
		\small{\begin{equation*}\label{eqn:conformal-via-pvalues_for_mondrian}Y_{n+1}\in \cC(X_{n+1}) \Longleftrightarrow \bar{p}:=\frac{1+\sum_{i\in[n],g(X_i,Y_i) = k}\ind{S_i\geq S_{n+1}}}{1+|\cI_k|} >  \alpha\end{equation*}}
		
		\item Next, fix any label $k\in[K]$ with $\P(g(X_{n+1},Y_{n+1})=k)>0$, and let $\cZ_0 = \{(x,y)\in\cX \times\cY: g(x,y)=k\} \subseteq \cX\times \cY$. By Lemma, the quantity
		\[p = \frac{1+\sum_{i\in[n]}\ind{(X_i,Y_i)\in\cZ_0,S_i\geq S_{n+1}}}{1+\sum_{i\in[n]}\ind{(X_i,Y_i)\in\cZ_0}} = \bar{p}\]
		satisfies $\P(p\leq \alpha\mid g(X_{n+1},Y_{n+1})=k)\leq \alpha$.\\ 
		This completes the proof.
	\end{itemize}
	
\end{frame}


\begin{frame}\frametitle{Another Relaxation on test-conditional coverage}
	\begin{itemize}
		\item Another idea relaxing the problem is to weaken the condition of $X$.
		\item New goal:
		$\label{eqn:test_conditional_relax_alpha_delta}\P\big(Y_{n+1}\in\cC(X_{n+1})\mid X_{n+1}\in\cX_0\big)\geq 1-\alpha\textnormal{ for all $P$, and all $\cX_0\subseteq\cX$ with $P_X(\cX_0)\geq \delta$}.$
		\item Trivial Solution : much more strict target "Marginal coverage level" and use randomization
		\begin{alertblock}{CP satisfying new goal}
			\begin{enumerate}
				\item Construct $\cC'(X_{n+1})$, using any method that guarantees marginal coverage at level $1-c\alpha\delta$.
				\item With probability $\frac{1-\alpha}{1-c\alpha}$, return $\cC(X_{n+1}) = \cC'(X_{n+1})$; otherwise, return $\cC(X_{n+1}) = \varnothing$.
			\end{enumerate}
		\end{alertblock}
	\end{itemize}
\end{frame}

\begin{frame}\frametitle{Prove of the previous method}
	\begin{itemize}
		\item By construction of the method, we have
		\begin{align*}
			&\P(Y_{n+1}\in\cC(X_{n+1})\mid X_{n+1}\in \cX_0) \\
			&=\frac{1-\alpha}{1-c\alpha} \P(Y_{n+1}\in\cC'(X_{n+1})\mid X_{n+1}\in \cX_0)
		\end{align*}
		\item Next
		\begin{align*}
			&\P(Y_{n+1}\not\in\cC'(X_{n+1})\mid X_{n+1}\in\cX_0) \\ &\le \delta^{-1} \P(Y_{n+1}\not\in\cC'(X_{n+1}))\indsub{X_{n+1}\in \cX_0} \le
			\delta^{-1} \cdot c\alpha \delta = c\alpha
		\end{align*}
		
		Combining these two calculations proves the result.
	\end{itemize}
\end{frame}

\begin{frame}\frametitle{Lower bound of the CI length in prev. method}
	\begin{block}{Definition : $L_P$}
		\small let $L_P(1-t)$ be the minimum length of any \emph{oracle} prediction interval  $\cC^P_{1-t}$ which is constructed given knowledge of the distribution $P$ (i.e., not a distribution-free method), and has coverage level $1-t$:
		\begin{align*}\label{eqn:L_P_def}L_P(1-t) = \inf\{ \E_P[\textnormal{Leb}(\cC^P_{1-t}(X))] \ : \\
			\textnormal{ $\cC^P_{1-t}$ satisfies }\P_P(Y\in \cC^P_{1-t}(X))\geq 1-t\}.\end{align*}
	\end{block}

	\begin{block}{Theorem 4.13. Lower Bound of CI-length of the $\cC$}
		Suppose $\cC$ satisfies the distribution-free relaxed test-conditional coverage condition, and let $\cY=\R$. Then, for any distribution $P$ on $\cX\times\R$ for which the marginal $P_X$ is nonatomic,
		\small\[\E[\textnormal{Leb}(\cC(X_{n+1}))] \geq \inf_{c\in[0,1]} \left\{\frac{1-\alpha}{1-c\alpha} \cdot L_P(1-c\alpha\delta)\right\}.\] 
	\end{block}
\end{frame}

\begin{frame}\frametitle{Lower bound of the CI length in prev. method (Cont.)}
	\begin{itemize}
		\item to prove the previous theorem, we need some lemmas.
	\end{itemize}
	\begin{block}{Lemma 4.14}
		\small Suppose $\cC$ satisfies the distribution-free relaxed test-conditional coverage condition. Let $P$ be any distribution on $\cX\times\cY$ s.t the marginal $P_X$ is nonatomic, then\\
		$\P_P\big(Y_{n+1}\in \cC(X_{n+1})\mid (X_{n+1},Y_{n+1})\in B\big) \geq 1-\alpha$\\
		for any $B\subseteq\cX\times\cY$ with $P(B)\geq \delta$
	\end{block}
\end{frame}
\begin{frame}\frametitle{Lower bound of the CI length in prev. method (Cont.)}
	\begin{itemize}
		\item to prove the previous theorem, we need some lemmas.
	\end{itemize}
	\begin{block}{Lemma 4.15. The sample–resample construction}
		\small Let $P$ be a distribution on $\cZ$, and let $m,M\geq 1$. Let $P^m$ denote the corresponding product distribution on $\cZ^m$---that is, the distribution of $(Z_1,\dots,Z_m)$, where $Z_1,\dots,Z_m\iidsim P$. Moreover, let $Q$ denote the distribution on $\cZ^m$ obtained by the following process to generate $(Z_1,\dots,Z_m)$:
		\begin{enumerate}
			\item Sample $Z^{(1)},\dots,Z^{(M)}\iidsim P$, and define the empirical distribution $\widehat{P}_M = \frac{1}{M}\sum_{i=1}^M \delta_{Z^{(i)}}$;
			\item Sample $Z_1,\dots,Z_m\iidsim \widehat{P}_M$.
		\end{enumerate}
		Then
		\[\dtv\big(P^m,Q\big) \leq \frac{m(m-1)}{2M},\]
		where $\dtv$ denotes the total variation distance between distributions.
	\end{block}
\end{frame}


\begin{frame}\frametitle{}
    \vfill
    \begin{center}
        \begin{Huge}Thank you!\end{Huge}
    \end{center}
    \vfill
\end{frame}

\begin{frame}[t, allowframebreaks]\frametitle{References}
    \nocite{*} % FIXME: this includes all references in ref.bib
    \bibliography{ref}
\end{frame}
\end{document}
