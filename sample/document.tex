\documentclass[11pt, a4paper]{article}

\usepackage{fancyhdr}
\usepackage[left=2cm, right=2cm, top=3cm, bottom=3cm]{geometry}
% \usepackage[utf8]{inputenc}
\usepackage{amsmath,amssymb}
\usepackage{enumitem}
\usepackage{booktabs}
\usepackage{xcolor}
\usepackage{kotex}

\usepackage{hyperref}
\definecolor{links}{rgb}{0.36,0.54,0.66}
\hypersetup{
   colorlinks = true,
    linkcolor = black,
     urlcolor = blue,
    citecolor = blue,
    filecolor = blue,
    pdfauthor = {Author},
     pdftitle = {Title},
   pdfsubject = {subject},
  pdfkeywords = {one, two},
  pdfproducer = {LaTeX},
   pdfcreator = {pdfLaTeX},
   }

\usepackage{enumitem}
\usepackage{pifont}

% enumi 값을 기반으로 ①, ②, ③...을 출력
\newcommand{\circlednum}[1]{\ding{\numexpr171+#1}} % ①=172

\newcommand{\PP}{\mathbb{P}}
\newcommand{\EE}{\mathbb{E}}
\newcommand\dis{\displaystyle}
\newcommand\var{\operatorname{Var}}

\pagestyle{fancy}
\lhead{\lecname}
\chead{\textbf{\doctitle}}
\rhead{\name}

\newcommand{\doctitle}{예시 도큐먼트 타입의 \LaTeX 문서}
\newcommand{\lecname}{2025 Fall CP seminar}
\newcommand{\name}{Sungwoo Park}
\newcommand{\todaydate}{Aug. 19, 2025}

\begin{document}

\title{\lecname \doctitle}
\author{\name}
\date{\todaydate}

\begin{flushright}
\lecname \\
\todaydate \\
\name \\
\end{flushright}

\begin{center}
\Large
\bfseries
\doctitle
\end{center}

% suppress the fancy header on the first page only
\thispagestyle{plain}

샘플로 작성하는 문서입니다.

\section{Introduction}
이 문서는 \LaTeX{} 문서 작성의 예시로 작성되었습니다. 이 문서에서는 다양한 수학 기호와 환경을 사용하여 내용을 표현합니다.

\subsection{문서 구조}
\[\sum e_i^2 \overset{(1)}{=}\sum(y_i - \hat{y_i})^2 \overset{(2)}{=} S_{xx}+\hat{\beta}^2S_{xx}-2\hat{\beta}S_{xy}\overset{(3)}{=}S_{yy}-\frac{S_{xy}^2}{S_{xx}}\]

\subsubsection{수학 기호}
다양한 수학 기호와 환경을 사용하여 내용을 표현합니다.

\subsection*{기타 사항}
별을 붙이면 번호가 매겨지지 않습니다.

\section{Loren Ipsum}
\subsubsection{예시 제목}
단순선형회귀모형 \(y_i = \alpha + \beta x_i + e_i\)에서 최소제곱법으로 구한 회귀식은 \(\hat{y}_i = \hat{\alpha} + \hat{\beta} x_i\) 이다. 여기서 \(\hat{\alpha} = \bar{y} - \hat{\beta}\bar{x}\) 이다. 오차제곱합 \(Q(\alpha, \beta) = \sum(y_i - \alpha - \beta x_i)^2\)을 최소화하는 \(\hat{\alpha}, \hat{\beta}\)는 다음 두 식을 만족한다.
\begin{align*}
    \frac{\partial Q}{\partial \alpha} \bigg|_{\hat{\alpha}, \hat{\beta}} = -2 \sum(y_i - \hat{\alpha} - \hat{\beta} x_i) = -2 \sum \hat{e}_i = 0 \\
    \frac{\partial Q}{\partial \beta} \bigg|_{\hat{\alpha}, \hat{\beta}} = -2 \sum x_i(y_i - \hat{\alpha} - \hat{\beta} x_i) = -2 \sum x_i \hat{e}_i = 0
\end{align*}

\subsubsection*{연습문제 8.4 단순선형회귀모형 에서의 성질}
\begin{enumerate}[label=(\alph*)]
    \item 8.3과 8.4의 \(c_i\) 정의가 다름에 유의하자.
    \[ \sum c_i y_i = \sum \frac{x_i - \bar{x}}{S_{xx}} y_i = \frac{1}{S_{xx}} \sum (x_i-\bar{x})y_i \]
    위 문제 8.3에서 보았듯이 \(\sum(x_i-\bar{x})y_i = S_{xy}\) 이므로,
    \[ \sum c_i y_i = \frac{S_{xy}}{S_{xx}} = \hat{\beta} \]
\end{enumerate}

\begin{enumerate}[label=(\alph*), start=3]
    \item
    \(\hat{\beta}\)는 \(Y_i\)들의 선형결합 \(\sum c_i Y_i\) 이므로 기댓값은,
    \[ E(\hat{\beta}) = E\left(\sum c_i Y_i\right) = \sum c_i E(Y_i) \]
    모형에서 \(E(Y_i) = \alpha + \beta x_i\) 이므로,
    \[ E(\hat{\beta}) = \sum c_i (\alpha + \beta x_i) = \alpha \sum c_i + \beta \sum c_i x_i \]
    여기서 \(\sum c_i\) 와 \(\sum c_i x_i\)를 계산하면,
    \[ \sum c_i = \sum \frac{x_i - \bar{x}}{S_{xx}} = \frac{1}{S_{xx}} \sum(x_i - \bar{x}) = 0 \]
    \[ \sum c_i x_i = \sum \frac{x_i - \bar{x}}{S_{xx}} x_i = \frac{1}{S_{xx}} \sum (x_i - \bar{x})x_i = \frac{1}{S_{xx}} \sum (x_i - \bar{x})(x_i - \bar{x}) = \frac{S_{xx}}{S_{xx}} = 1 \]
    따라서, \(E(\hat{\beta}) = \alpha(0) + \beta(1) = \beta\). 즉 \(\hat{\beta}\)는 \(\beta\)의 불편추정량이다.

    \item
    \(Y_i\)들이 서로 독립이고 \(Var(Y_i) = \sigma^2\) 이므로,
    \[ Var(\hat{\beta}) = Var\left(\sum c_i Y_i\right) = \sum c_i^2 Var(Y_i) = \sigma^2 \sum c_i^2 \]
    \(\sum c_i^2\)를 계산하면,
    \[ \sum c_i^2 = \sum \left(\frac{x_i - \bar{x}}{S_{xx}}\right)^2 = \frac{1}{S_{xx}^2} \sum(x_i - \bar{x})^2 = \frac{S_{xx}}{S_{xx}^2} = \frac{1}{S_{xx}} \]
    따라서, \(\dis Var(\hat{\beta}) = \sigma^2 \left(\frac{1}{S_{xx}}\right) = \frac{\sigma^2}{S_{xx}}\) \hfill $\blacksquare$
\end{enumerate}


\end{document}